% Options for packages loaded elsewhere
\PassOptionsToPackage{unicode}{hyperref}
\PassOptionsToPackage{hyphens}{url}
%
\documentclass[
]{article}
\usepackage{lmodern}
\usepackage{amssymb,amsmath}
\usepackage{ifxetex,ifluatex}
\ifnum 0\ifxetex 1\fi\ifluatex 1\fi=0 % if pdftex
  \usepackage[T1]{fontenc}
  \usepackage[utf8]{inputenc}
  \usepackage{textcomp} % provide euro and other symbols
\else % if luatex or xetex
  \usepackage{unicode-math}
  \defaultfontfeatures{Scale=MatchLowercase}
  \defaultfontfeatures[\rmfamily]{Ligatures=TeX,Scale=1}
\fi
% Use upquote if available, for straight quotes in verbatim environments
\IfFileExists{upquote.sty}{\usepackage{upquote}}{}
\IfFileExists{microtype.sty}{% use microtype if available
  \usepackage[]{microtype}
  \UseMicrotypeSet[protrusion]{basicmath} % disable protrusion for tt fonts
}{}
\makeatletter
\@ifundefined{KOMAClassName}{% if non-KOMA class
  \IfFileExists{parskip.sty}{%
    \usepackage{parskip}
  }{% else
    \setlength{\parindent}{0pt}
    \setlength{\parskip}{6pt plus 2pt minus 1pt}}
}{% if KOMA class
  \KOMAoptions{parskip=half}}
\makeatother
\usepackage{xcolor}
\IfFileExists{xurl.sty}{\usepackage{xurl}}{} % add URL line breaks if available
\IfFileExists{bookmark.sty}{\usepackage{bookmark}}{\usepackage{hyperref}}
\hypersetup{
  pdftitle={Analysis},
  hidelinks,
  pdfcreator={LaTeX via pandoc}}
\urlstyle{same} % disable monospaced font for URLs
\usepackage[margin=1in]{geometry}
\usepackage{color}
\usepackage{fancyvrb}
\newcommand{\VerbBar}{|}
\newcommand{\VERB}{\Verb[commandchars=\\\{\}]}
\DefineVerbatimEnvironment{Highlighting}{Verbatim}{commandchars=\\\{\}}
% Add ',fontsize=\small' for more characters per line
\usepackage{framed}
\definecolor{shadecolor}{RGB}{248,248,248}
\newenvironment{Shaded}{\begin{snugshade}}{\end{snugshade}}
\newcommand{\AlertTok}[1]{\textcolor[rgb]{0.94,0.16,0.16}{#1}}
\newcommand{\AnnotationTok}[1]{\textcolor[rgb]{0.56,0.35,0.01}{\textbf{\textit{#1}}}}
\newcommand{\AttributeTok}[1]{\textcolor[rgb]{0.77,0.63,0.00}{#1}}
\newcommand{\BaseNTok}[1]{\textcolor[rgb]{0.00,0.00,0.81}{#1}}
\newcommand{\BuiltInTok}[1]{#1}
\newcommand{\CharTok}[1]{\textcolor[rgb]{0.31,0.60,0.02}{#1}}
\newcommand{\CommentTok}[1]{\textcolor[rgb]{0.56,0.35,0.01}{\textit{#1}}}
\newcommand{\CommentVarTok}[1]{\textcolor[rgb]{0.56,0.35,0.01}{\textbf{\textit{#1}}}}
\newcommand{\ConstantTok}[1]{\textcolor[rgb]{0.00,0.00,0.00}{#1}}
\newcommand{\ControlFlowTok}[1]{\textcolor[rgb]{0.13,0.29,0.53}{\textbf{#1}}}
\newcommand{\DataTypeTok}[1]{\textcolor[rgb]{0.13,0.29,0.53}{#1}}
\newcommand{\DecValTok}[1]{\textcolor[rgb]{0.00,0.00,0.81}{#1}}
\newcommand{\DocumentationTok}[1]{\textcolor[rgb]{0.56,0.35,0.01}{\textbf{\textit{#1}}}}
\newcommand{\ErrorTok}[1]{\textcolor[rgb]{0.64,0.00,0.00}{\textbf{#1}}}
\newcommand{\ExtensionTok}[1]{#1}
\newcommand{\FloatTok}[1]{\textcolor[rgb]{0.00,0.00,0.81}{#1}}
\newcommand{\FunctionTok}[1]{\textcolor[rgb]{0.00,0.00,0.00}{#1}}
\newcommand{\ImportTok}[1]{#1}
\newcommand{\InformationTok}[1]{\textcolor[rgb]{0.56,0.35,0.01}{\textbf{\textit{#1}}}}
\newcommand{\KeywordTok}[1]{\textcolor[rgb]{0.13,0.29,0.53}{\textbf{#1}}}
\newcommand{\NormalTok}[1]{#1}
\newcommand{\OperatorTok}[1]{\textcolor[rgb]{0.81,0.36,0.00}{\textbf{#1}}}
\newcommand{\OtherTok}[1]{\textcolor[rgb]{0.56,0.35,0.01}{#1}}
\newcommand{\PreprocessorTok}[1]{\textcolor[rgb]{0.56,0.35,0.01}{\textit{#1}}}
\newcommand{\RegionMarkerTok}[1]{#1}
\newcommand{\SpecialCharTok}[1]{\textcolor[rgb]{0.00,0.00,0.00}{#1}}
\newcommand{\SpecialStringTok}[1]{\textcolor[rgb]{0.31,0.60,0.02}{#1}}
\newcommand{\StringTok}[1]{\textcolor[rgb]{0.31,0.60,0.02}{#1}}
\newcommand{\VariableTok}[1]{\textcolor[rgb]{0.00,0.00,0.00}{#1}}
\newcommand{\VerbatimStringTok}[1]{\textcolor[rgb]{0.31,0.60,0.02}{#1}}
\newcommand{\WarningTok}[1]{\textcolor[rgb]{0.56,0.35,0.01}{\textbf{\textit{#1}}}}
\usepackage{graphicx,grffile}
\makeatletter
\def\maxwidth{\ifdim\Gin@nat@width>\linewidth\linewidth\else\Gin@nat@width\fi}
\def\maxheight{\ifdim\Gin@nat@height>\textheight\textheight\else\Gin@nat@height\fi}
\makeatother
% Scale images if necessary, so that they will not overflow the page
% margins by default, and it is still possible to overwrite the defaults
% using explicit options in \includegraphics[width, height, ...]{}
\setkeys{Gin}{width=\maxwidth,height=\maxheight,keepaspectratio}
% Set default figure placement to htbp
\makeatletter
\def\fps@figure{htbp}
\makeatother
\setlength{\emergencystretch}{3em} % prevent overfull lines
\providecommand{\tightlist}{%
  \setlength{\itemsep}{0pt}\setlength{\parskip}{0pt}}
\setcounter{secnumdepth}{-\maxdimen} % remove section numbering

\title{Analysis}
\author{}
\date{\vspace{-2.5em}}

\begin{document}
\maketitle

{
\setcounter{tocdepth}{3}
\tableofcontents
}
\hypertarget{objectives}{%
\subsubsection{Objectives}\label{objectives}}

A short description of the projects main problem is available on our
\href{index.html}{home page}. We are tasked with writing a simulation
that choose large n and m that we can sample the expected travel time of
m-n+1 circuits which we are asked to show is approximately the true
expected travel time of a circuit chosen uniformly
\[ \sum_{i} f(i)\mu(i) \approx \frac{1}{m+1} \sum_{k=0}^{m} f(X_k)\].
Towards this goal we first labeled 20 cities numerically from 1:20

\begin{Shaded}
\begin{Highlighting}[]
\NormalTok{U <-}\KeywordTok{c}\NormalTok{(}\DecValTok{1}\OperatorTok{:}\DecValTok{20}\NormalTok{) }\CommentTok{#vertices}
\end{Highlighting}
\end{Shaded}

In graph terminology these are our vertices and will make up the
elements within our state space. Originally we had proceeded with the
idea that we can sample from a generator matrix that would simulate all
the Hamiltonian circuits but this would require non-existent computing
power to generate a \(20!\) by \(20!\) matrix.

Thus its best to consider the state space as a collection of all the
valid circuits of which there are

\begin{verbatim}
## [1] 2.432902e+18
\end{verbatim}

Thus the generator isn't a matrix but instead a method which will choose
two cities uniformly at random. Before writing the method, we first
construct the edges as per the requirement. To this end we first
constructed a \(20\) by \(20\) distance matrix D and populated it using
a uniform(0,1) density.

\hypertarget{distance-matrix}{%
\subsubsection{Distance Matrix}\label{distance-matrix}}

\begin{Shaded}
\begin{Highlighting}[]
\NormalTok{D <-}\StringTok{ }\KeywordTok{matrix}\NormalTok{(}\KeywordTok{runif}\NormalTok{(}\DataTypeTok{n=}\DecValTok{20}\OperatorTok{*}\DecValTok{20}\NormalTok{, }\DataTypeTok{min=}\DecValTok{0}\NormalTok{, }\DataTypeTok{max =}\DecValTok{1}\NormalTok{), }\DataTypeTok{ncol =} \DecValTok{20}\NormalTok{) }\CommentTok{#uniformly distributed distance matrix}
\end{Highlighting}
\end{Shaded}

We then manually specified 5 pairs (i,j) for which the entry in the
distance matrix D would be zero implying travel restriction.We then
transform D into a symmetric matrix by assigning the lower triangular of
the matrix the entries of the upper triangular of the transpose of
D.Finally we set the diagonal entries to zero as its assumed the
distance from city i to i is 0. Thus as you can see there are 185 pairs
with positive densities (19*20)/2 - 5 from the matrix or
\({20 \choose 2}-5\).

\begin{Shaded}
\begin{Highlighting}[]
\NormalTok{D[}\DecValTok{4}\NormalTok{,}\DecValTok{6}\NormalTok{] =}\StringTok{ }\DecValTok{0}
\NormalTok{D[}\DecValTok{9}\NormalTok{,}\DecValTok{13}\NormalTok{]=}\StringTok{ }\DecValTok{0}
\NormalTok{D[}\DecValTok{1}\NormalTok{,}\DecValTok{19}\NormalTok{]=}\StringTok{ }\DecValTok{0} 
\NormalTok{D[}\DecValTok{10}\NormalTok{,}\DecValTok{17}\NormalTok{]=}\StringTok{ }\DecValTok{0}  
\NormalTok{D[}\DecValTok{3}\NormalTok{,}\DecValTok{8}\NormalTok{] =}\StringTok{ }\DecValTok{0} \CommentTok{#NO TRAVEL NO DISTANCE}
\NormalTok{D[}\KeywordTok{lower.tri}\NormalTok{(D)] =}\StringTok{ }\KeywordTok{t}\NormalTok{(D)[}\KeywordTok{lower.tri}\NormalTok{(D)] }\CommentTok{#symmetric matrix}
\KeywordTok{diag}\NormalTok{(D) <-}\StringTok{ }\KeywordTok{rep}\NormalTok{(}\DecValTok{0}\NormalTok{,}\DecValTok{20}\NormalTok{) }\CommentTok{#diagonals are 0}
\end{Highlighting}
\end{Shaded}

We note that the forbidden pairs must be entered in the form of
D{[}i,j{]} where i\textless j or else when we go to make the matrix
symmetric the upper triangular entry is still most likely non-zero. This
can be remedied by manually make all 10 entries 0, or letting
j\textless i and assigning the upper-triangular of D with the lower
triangular of the transpose of D or in a proper application handling
user input.

\hypertarget{algorithm}{%
\subsubsection{Algorithm}\label{algorithm}}

Thus with the vertices set up and the weighted-edges specified, we have
initialized our graph. We can now consider our \(Q\). As mentioned
above, the most prudent method as suggested was a method which picks two
points uniformly at random and intechanges them. The code chunck below
is the function that we desire.

\begin{Shaded}
\begin{Highlighting}[]
\NormalTok{tspMCMC <-}\StringTok{ }\ControlFlowTok{function}\NormalTok{(n,m)\{}
\NormalTok{S <-}\StringTok{ }\KeywordTok{list}\NormalTok{(}\KeywordTok{c}\NormalTok{(}\DecValTok{1}\OperatorTok{:}\DecValTok{20}\NormalTok{)) }\CommentTok{#preferably this sequence will be randomly generated and a proper circuit.}
\NormalTok{rejection_counter <-}\StringTok{ }\DecValTok{0}
\NormalTok{i <-}\StringTok{ }\DecValTok{1}
\NormalTok{j <-}\StringTok{ }\DecValTok{1}
\NormalTok{travel_time <-}\StringTok{ }\KeywordTok{rep}\NormalTok{(}\DecValTok{0}\NormalTok{, m}\OperatorTok{-}\NormalTok{n}\OperatorTok{+}\DecValTok{1}\NormalTok{)}
\ControlFlowTok{while}\NormalTok{(i }\OperatorTok{<=}\StringTok{ }\NormalTok{m)\{}
\NormalTok{  P <-}\StringTok{ }\KeywordTok{sample}\NormalTok{(}\DecValTok{1}\OperatorTok{:}\DecValTok{20}\NormalTok{,}\DataTypeTok{size=}\DecValTok{1}\NormalTok{) }\CommentTok{#basic sampling }
\NormalTok{  Q <-}\StringTok{ }\KeywordTok{sample}\NormalTok{(}\DecValTok{1}\OperatorTok{:}\DecValTok{20}\NormalTok{,}\DataTypeTok{size=}\DecValTok{1}\NormalTok{) }\CommentTok{#likewise for proposed state, additionally Q <- sample(1:20,size=1, prob =D[P,]) removes need for indicatorPermissable}
\NormalTok{  index_p =}\StringTok{ }\KeywordTok{match}\NormalTok{(P, }\KeywordTok{unlist}\NormalTok{(S[i])) }\CommentTok{#the position of P in the sequence}
\NormalTok{  index_q =}\StringTok{ }\KeywordTok{match}\NormalTok{(Q, }\KeywordTok{unlist}\NormalTok{(S[i])) }\CommentTok{#likewise for the second element in the pair}
\NormalTok{  k <-}\StringTok{ }\KeywordTok{unlist}\NormalTok{(S[i])}
  \ControlFlowTok{if}\NormalTok{(}\KeywordTok{indicatorPermissible}\NormalTok{(k,index_p,index_q) }\OperatorTok{==}\StringTok{ }\DecValTok{1}\NormalTok{)\{}
    \ControlFlowTok{if}\NormalTok{(i }\OperatorTok{>=}\StringTok{ }\NormalTok{n)\{}
\NormalTok{      travel_time[j] =}\StringTok{ }\KeywordTok{distance_travelled}\NormalTok{(S[i])}
\NormalTok{      j <-}\StringTok{ }\NormalTok{j }\OperatorTok{+}\StringTok{ }\DecValTok{1}
\NormalTok{    \}}
\NormalTok{    S[i}\OperatorTok{+}\DecValTok{1}\NormalTok{] <-}\StringTok{ }\KeywordTok{list}\NormalTok{(}\KeywordTok{c}\NormalTok{(}\KeywordTok{replace}\NormalTok{(k, }\KeywordTok{c}\NormalTok{(index_p,index_q), }\KeywordTok{c}\NormalTok{(Q,P)))) }\CommentTok{#adds a new sequence to S interchanging P and Q }
\NormalTok{    i <-}\StringTok{ }\NormalTok{i }\OperatorTok{+}\StringTok{ }\DecValTok{1}
    \CommentTok{#return(S[i+1])}
\NormalTok{  \}}
  \ControlFlowTok{else}\NormalTok{\{}
\NormalTok{    rejection_counter =}\StringTok{ }\NormalTok{rejection_counter}\OperatorTok{+}\DecValTok{1} \CommentTok{#see how many time our proposed transition introduced forbidden travel}
\NormalTok{  \}}
\NormalTok{\}}
\NormalTok{   answer =}\StringTok{ }\KeywordTok{mean}\NormalTok{(travel_time)}\OperatorTok{*}\NormalTok{(m}\OperatorTok{-}\NormalTok{n}\OperatorTok{+}\DecValTok{1}\NormalTok{)}\OperatorTok{/}\NormalTok{(m}\OperatorTok{+}\DecValTok{1}\NormalTok{)}
   \KeywordTok{print}\NormalTok{(}\KeywordTok{length}\NormalTok{(travel_time))}
   \KeywordTok{print}\NormalTok{(rejection_counter)}
   \KeywordTok{return}\NormalTok{(answer)}
\NormalTok{\}}
\end{Highlighting}
\end{Shaded}

There is alot to unpack but very concisely it takes two arguments
n(integer) and m(integer). It then samples two points, extracts their
place in the sequence and pass it to along with the sequence to another
function called \emph{indicatorPermissable.} It's the responsibility of
the \emph{indicatorPermissable} to return 0 if the proposed interchange
creates a sequence where travel between two forbidden cities is
introduced. In the case where the indicator returns 1 we accept the new
sequence and add it to the list of valid circuits that we have
sampled.Furthermore if we have already sampled n \textbf{valid} circuits
we also start computing the travel times by passing the sequence to a
function called \emph{distance\_travelled}.

One decision of importance that was taken was the use of a while loop
instead of a for loop. This is done for multiple reasons one of which is
that it gives up greater control of our simulation. It allows us to
sample m unique circuits without repetitions. For example, if there is a
proposed interchange that is rejected we don't add it to the list of
valid sequences we have sampled. This is done so that we don't have to
compute the travel time for a sequence multiple times in the case where
we have rejected an interchange.

\hypertarget{observations}{%
\subsubsection{Observations}\label{observations}}

Now let us test the algorithm using the same forbidden cities from above
and a Uni(0,1) distributed distance matrix, additionally choose
\(n=20,000\) and \(m=40,000\)

\begin{Shaded}
\begin{Highlighting}[]
\KeywordTok{tspMCMC}\NormalTok{(}\DecValTok{20000}\NormalTok{,}\DecValTok{40000}\NormalTok{)}
\end{Highlighting}
\end{Shaded}

\begin{verbatim}
## Rejection Counter:  12033 
## Average Return Time: 5.135202 seconds 
## Computational Length: 2.0703 seconds
\end{verbatim}

Furthermore keeping the distance matrix fixed we can observe the results
for multiple choices of n and m.

\begin{Shaded}
\begin{Highlighting}[]
\KeywordTok{tspMCMC}\NormalTok{(}\DecValTok{20000}\NormalTok{,}\DecValTok{40000}\NormalTok{)}
\KeywordTok{tspMCMC}\NormalTok{(}\DecValTok{40000}\NormalTok{,}\DecValTok{80000}\NormalTok{)}
\KeywordTok{tspMCMC}\NormalTok{(}\DecValTok{80000}\NormalTok{,}\DecValTok{160000}\NormalTok{)}
\KeywordTok{tspMCMC}\NormalTok{(}\DecValTok{160000}\NormalTok{,}\DecValTok{320000}\NormalTok{)}
\end{Highlighting}
\end{Shaded}

\begin{verbatim}
## tspMCM( 20000 , 40000 ) 
## Rejection Counter:  11994 
## Average Return Time: 4.727875 seconds
\end{verbatim}

\begin{verbatim}
## tspMCM( 40000 , 80000 ) 
## Rejection Counter:  24100 
## Average Return Time: 4.716286 seconds
\end{verbatim}

\begin{verbatim}
## tspMCM( 80000 , 160000 ) 
## Rejection Counter:  48129 
## Average Return Time: 4.712535 seconds
\end{verbatim}

\begin{verbatim}
## tspMCM( 160000 , 320000 ) 
## Rejection Counter:  97288 
## Average Return Time: 4.709745 seconds
\end{verbatim}

\begin{verbatim}
## tspMCM( 1e+06 , 2e+06 ) 
## Rejection Counter:  604861 
## Average Return Time: 4.711487 seconds
\end{verbatim}

\begin{verbatim}
## Computational Length: 2.792831 minutes
\end{verbatim}

\hypertarget{appendix}{%
\subsubsection{Appendix}\label{appendix}}

\end{document}
